\chapter{Introduction}  \label{intro} 
Spectroscopy is one of the most powerful tools that is used by scientists to understand the structure and properties of physical systems. In chemistry, spectroscopy is applied for obtaining characteristic quantities of matter and its chemical environment. An electromagnetic radiation source can be set up in the experiments, and then the impact of the radiation on the system is measured as a function of wavelength and frequency.\cite{Svanberg2023} As for the theoretical counterpart, quantum chemical methods based on the Schr\"odinger equation are developed for calculating the properties of excited states or response properties induced by the external electromagnetic field, which is long-established and prevalent in the field.\cite{Szabo2012, Helgaker2013} They bridge, validate and guide the experimental studies. Interesting and spectacular up-to-date applications including but not limited to: utilizing UV absorption spectra calculated with density functional theory (DFT) for the determination of Molnupiravir, a drug developed for mild COVID-19, in a more environmental friendly pharmaceutical preparation;\cite{Abdelazim2023} predicting interstellar molecules guided by a coupled cluster (CC) calculation of rotational spectra for astronomical searches;\cite{Puzzarini2023} rapid classification of tea products incorporating configuration interaction (CI) calculations with X-ray photoelectron spectroscopy (XPS) measured in experiments.\cite{Jiang2022} The development of an efficient and accurate method for calculating such properties is always highly in need, especially for higher-order properties such as specific rotation that are much more demanding in terms of computational resources compared to energies or dipole moments.\cite{Pecul2005, Crawford2007, Bohle2021}

For quantum chemical calculations, solving the Schr\"odinger equation is essential. However, obtaining the exact solution is only feasible for the simplest systems. The Hartree-Fock (HF)\cite{Slater1951, Szabo2012} approximation of the wave function is one of the most fundamental in computational chemistry. The essence of the Hartree-Fock approximation lies in representing the many-electron wave function as a single determinant composed of molecular orbitals, known as a Slater determinant. This single configuration state function is commonly utilized as a reference state for various higher levels of theory. Other more accurate approaches based upon HF are called post Hartree-Fock (post-HF) methods accordingly which include electron-electron correlation that is beyond the Slater determinant. For example, the configuration interaction (CI) method\cite{Sherrill1999} represents the exact wave function as a linear combination of Slater determinants so that all the electron configurations can be included with a complete basis (the number of N-electron functions in the basis set is infinite). Although full CI can give the exact solution of the N-electron problem formally, it is often extremely expensive whereas truncated CI is not size-consistent or size-extensive leading to difficulties in studying larger systems. Another important approximation for the correlation energy is coupled cluster (CC) theory.\cite{Crawford2000} Different from CI, the coupled cluster operator applies to the reference wave function in an exponential form. This unique exponential \textit{ansatz} gives rise to the most important characteristics of coupled cluster, size-consistency and further, size-extensivity. CC methods are systematically improvable and in practice, they are often used in truncated form. Coupled cluster singles and doubles with a perturbed triples correction (CCSD(T))\cite{Purvis1982} is referred to the gold standard. Other truncated CC methods include CCD, CCSD, CC2,\cite{Christiansen1995} CC3,\cite{Koch1997} CCSDT, etc. In this work, CC serves the role of the wave function approximation binding with the approaches targeting response properties. Since CC methods are the substantial part of the foundation and the objectives in this work, the basis and working equations that are used in Chapters 3 and 4 will be presented in Chapter 2. 

For the response properties specifically, response theory and real-time methods are two prominent approaches. Molecular properties induced by frequency-dependent external fields must naturally be calculated with a time-dependent approach. In 1928, Rosenfeld first published time-dependent perturbation methods and defined two optical activity tensors (also known as Rosenfeld tensors) that are related to the polarizability and the specific rotation respectively, laying the foundations of subsequent studies.\cite{Rosenfeld1929} To avoid the summation over all the excited states involved in the formula of Rosenfeld tensors, linear response theory (LR)\cite{Olsen1985, Sekino1984} becomes a more practical and conventional approach with the excited states replaced by the response of the ground state. The frequency-dependent response functions for the optical activity tensors are derived starting from a time-dependent perturbed Hamiltonian. Response theory agrees with the experimental data very well, especially for spectroscopic properties.\cite{Kobayashi1994, Roos1996, Coriani2012} Its root in perturbation theory, however, limits its applications with strong external fields. In Chapter 4, LR, particularly LR-CC results are used as the reference values in the examination of the accuracy of the proposed novel approaches where weak interactions between the molecules and the fields are involved. The fundamentals will be included in Chapter 2. 

Real-time methods, on the other hand, are non-perturbative approaches that can cope with strong external fields beyond perturbation theory.\cite{Goings2018, Li2020} With an explicit time-dependence, properties are calculated in the time-domain by propagating wave function parameters based on the time-dependent Schr\"odinger equation (TDSE). The properties can further be transformed into the full frequency-domain by means of, for example, Fourier transform (FT) without a designated narrow frequency range. Other advantages including the natural connection to the experiments, the potential of combining advanced techniques in other fields such as laser spectroscopy and numerical integration are also key drivers of the development of real-time methods. As mentioned earlier, this work will focus on implementations within CC approximation. For real-time coupled cluster (RT-CC) methods, the challenge lies in two aspects: (1) The long propagation time for high-resolution spectra and the small step size for the adequate numerical stability of the simulation; (2) The balance between better accuracy with higher levels of theory and the high computational cost stemming from the polynomial scaling of CC methods. As a guide of general RT-CC simulations, the derivation of the differential equations of the coupled cluster amplitudes and the consideration of the external field and resulting properties are included in Chapter 2. 

Succeeding studies that seek for the strategies to accelerate the simulations are shown in Chapter 3 addressing the problem in computational and numerical aspects. Three perspectives are given: investigating the capability of the single-precision arithmetic compared to the conventional double-precision in quantum chemistry for the properties of interest; migrating the calculation from CPUs to GPUs to speed up the tensor contraction which is the major component of the working equations and most expensive part of the calculation; adopting and tailoring adaptive numerical integrators for RT-CC methods in order to lower the computational cost and improve the numerical stability particularly considering the high-intensity external fields. Chapter 4, instead, mainly presents the derivation and applications of real-time approximate coupled cluster singles, doubles and triples (RT-CC3) model with the singles amplitudes treated as zeroth-order and performing a function of  approximate orbital relaxation parameters. In addition to the absorption spectrum, polarizability, first hyperpolarizability and $G'$ tensor that is related to the optical rotation are also extracted from the real-time simulation with finite-difference methods.\cite{Ding2013} Results are compared to other RT-CC methods. Notably, a comparison with real-time (RT) time-dependent nonorthogonal orbital-optimized coupled cluster doubles (TDNOCCD)\cite{Krylov2000, Pedersen2001, Kristiansen2020} is carried out to explore more information about the mechanism and importance of orbital relaxation in real-time methods. A summary and conclusion section is given in Chapter 5. 