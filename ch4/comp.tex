\section{Computational Details} \label{comp_cc3} 
When calculating the absorption spectrum and comparing it with the EOM-CC results, an isotropic electric field shaped as a Gaussian function is applied to the system and shown as
\begin{equation}
\textbf{E}(t) = \mathcal{E}e^{-\frac{(t-\nu)^{2}}{2\sigma^{2}}}\textbf{n},
\end{equation}
where the vector $\textbf{n}$ represents the direction of the field as 
\begin{equation}
\textbf{n} = \frac{1}{\sqrt{3}}(\hat{i}+\hat{j}+\hat{k}).
\end{equation}
The center $\nu$ and the width $\sigma$ of the field are chosen to be 0.01 au and 0.001 au, respectively, to mimic a delta pulse that is switched on at the beginning of the propagation. For the calculation of RT-CC3/cc-pVDZ absorption spectrum of \ch{H_{2}O}, the field strength $\mathcal{E}$, step size $h$, and propagation time $t_{f}$ were chosen to be 0.01 au, 0.01 au, and 300 au, respectively. For the RT-CCSD/STO-3G absorption spectra of \ch{H_{2}}, \ch{LiH}, and \ch{C_{2}H_{4}}, the field strength $\mathcal{E}$, step size $h$, and propagation time $t_{f}$ were chosen to be 0.01 au, 0.01 au and 500 au, respectively.

To further test the performance of our RT-CC3 implementation, CPU and GPU calculations were carried out using water monomer, dimer, and trimer systems in both single- and double-precision. Each CPU calculation was run on a single node with an AMD EPYC 7702 chip, and each GPU calculation was run on a single node with an Nvidia Tesla P100 GPU. Tensor manipulation was conducted using NumPy\cite{Harris2020} and PyTorch\cite{Paszke2019} for the CPU and GPU calculations, respectively, with similar syntax. Tensor contraction was performed using {\tt opt\_einsum},\cite{Smith2018} and a PyTorch backend was specifically employed for the GPU calculation. All calculations kept the $1s$ orbitals of the oxygen atoms frozen.

For calculating dynamic polarizabilities and first hyperpolarizabilities, a set of RT calculations for the water molecule with the cc-pVDZ basis set\cite{Dunning1989} were executed using field strengths of $0.002\ \text{au}$, $-0.002\ \text{au}$, $0.004\ \text{au}$, and $-0.004\ \text{au}$ at both the CC3 and CCSD levels. The step size was set to $0.01\ \text{au}$. The carrier frequency of the field was set to $0.078\ \text{au}$, which corresponds to a wavelength of $582\ \text{nm}$ and is lower than the resonance at $0.247\ \text{au}$ for the water molecule. The molecule was subjected to a field in the $x$, $y$, and $z$ directions individually to obtain the corresponding elements of the polarizabilities and first hyperpolarizabilities tensors. For the $G'$ tensors, the same electric field was applied to the H$_{2}$ dimer with the cc-pVDZ basis set. The $G'$ tensor elements were calculated from the induced magnetic dipole moments. The frequency of $0.078\ \text{au}$ is below the resonance at $0.367\ \text{au}$. Curve fitting was performed using {\tt scipy.optimize.curve\_fit}.\cite{Virtanen2020} All calculations were performed on a single Nvidia Tesla P100 GPU, and both single- and double-precision calculations were conducted and compared. The results from the RT simulations were also compared to reference values obtained from the Psi4\cite{Smith2020} and CFOUR\cite{Matthews2020} packages. 

RT-CC methods were also compared to the time-dependent nonorthogonal orbital-optimized coupled cluster doubles (TDNOCCD) method\cite{Pedersen2001} for calculating polarizabilities with ten-electron systems including \ch{Ne}, \ch{HF}, \ch{H_{2}O}, \ch{NH_{3}}, and \ch{CH_{4}}. The field strengths of the propagations were chosen to be $0.001\ \text{au}$, $-0.001\ \text{au}$, $0.002\ \text{au}$, and $-0.002\ \text{au}$. Various frequencies below the resonance of the corresponding molecule were tested. The basis set was chosen to be aug-cc-pVDZ\cite{Woon1993} for \ch{HF}, \ch{H_{2}O}, \ch{NH_{3}}, and \ch{CH_{4}}, and d-aug-cc-pVDZ\cite{Woon1994} for \ch{Ne}. The QRCW method was utilized for accuracy and efficiency. The length of each propagation is two optical cycles, depending on the frequency of the field. The time step of the propagations was $0.01\ \text{au}$. All calculations were performed in double-precision. 

Electron dynamics in response to an applied electric field were modeled using RT-CCSD. The test cases included \ch{H_{2}}, \ch{LiH} and \ch{C_{2}H_{4}} with the STO-3G basis set.\cite{Pietro1983} The applied field is defined as
\begin{equation}
\textbf{E}(t) = \mathcal{E}sin(\omega t)\textbf{n}.
\end{equation}
For \ch{H_{2}}, the field was applied in the direction of the H-H bond to simulate Rabi oscillation. In the simulations of LiH, an isotropic field was applied for one optical cycle with various field strengths. The absorption spectra were calculated using the induced dipole moment after the field was turned off. The $1s$ orbital in the lithium atom was frozen. To investigate the Rabi oscillation of an approximate two-level system, namely the $\pi$-$\pi^{*}$ transition in \ch{C_{2}H_{4}}, the field was applied in the direction of the \ch{C}=\ch{C} double bond. The $1s$ orbitals of the carbon atoms were frozen. The step size of the calculations was chosen to be $0.01\ \text{au}$. The calculations were performed on a single Nvidia Tesla P100 GPU in double-precision.

All calculations were run in PyCC\cite{pycc} with the stationary electric and magnetic dipole operators extracted from Psi4. Runge-Kutta fourth-order integrator\cite{Butcher1996} was used for the RT propagations. For the series of water clusters, (H$_2$O)$_n$ up to $n=4$, used in section~\ref{results-cc3-1}, including \ch{H_{2}O} in section~\ref{results-cc3-22}, the coordinates were provided by Pokhilko et al.\cite{Pokhilko2018} Five ten-electron systems, \ch{Ne}, \ch{HF}, \ch{H_{2}O}, \ch{NH_{3}}, and \ch{CH_{4}}, were taken as test cases in section~\ref{results-cc3-22}, using coordinates provided by Kristiansen et al.\cite{Kristiansen2022} The coordinate of the \ch{H_{2}} dimer for the $G'$ tensor calculation in section~\ref{results-cc3-23} can be found in the dictionary of molecular structures of PyCC. In section~\ref{results-cc3-3}, the coordinate of \ch{H_{2}} and \ch{C_{2}H_{4}} were taken from Ref.~\citenum{Habenicht2014}, and the coordinate of \ch{LiH} was obtained from Ref.~\citenum{Provorse2015}.

