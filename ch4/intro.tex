\section{Introduction} \label{ch4-intro}
Coupled cluster (CC) theory\cite{Purvis1982, Crawford2000} has been proven to be one of the most accurate and robust correlated methods. The non-truncated CC method that contains all levels of excitations is equivalent to full configurational interaction (CI)\cite{Sherrill1999} and recovers the exact wave function. However, only truncated CC methods are feasible in practical applications, while they still suffer from polynomial scaling. For instance, CCS, CCSD,\cite{Purvis1982} CCSDT,\cite{Noga1987, Scuseria1988} and CCSDTQ\cite{Kucharski1998} scale at $N^{4}$, $N^{6}$, $N^{8}$, and $N^{10}$, respectively. 

Although the inclusion of singles and doubles has been widely used and affirmed to be effective and efficient, a higher level of theory is still in demand to achieve a closer comparison to full CI and/or recover missing properties. Covering triple excitation operators is an obvious starting point. It needs to be pointed out that triple excitations can already be obtained from lower excitation operators in CC by definition. For instance, the coupled cluster singles and doubles (CCSD) method can introduce triple excitations by having terms like $\hat{T}_{1}\hat{T}_{2}$ and $\hat{T}_{1}^{3}$. It is important to figure out which are the leading terms among all triple excitations and wether the addition of the triple excitation operator $\hat{T}_{3}$ is key to the problem, as it indeed impacts the computational efficiency. Many-body perturbation theory (MBPT)\cite{Bartlett1981} is a paramount and straightforward technique to show how higher excitations affect the electron-correlation effects and properties. According to MBPT, $\hat{T}_{3}$ contributes to and appears for the first time in the fourth-order correlation energy, while $\hat{T}_{1}\hat{T}_{2}$ and $\hat{T}_{1}^{3}$ first appear in higher orders. Thus, to make a method correct to the fourth order, triples are essential, not only for the correlation energy itself,\cite{Lee1984} but also for other properties, including excitation energies,\cite{Christiansen1995CC3, Watts1995, Christiansen1996, Kucharski2001} polarizabilities,\cite{Christiansen1998triple, Hald2003} NMR parameters,\cite{Gauss1996, Faber2017, Jaszunski2020} and so on.

Over the decades, alternatives to treat triples have been explored, as the full CCSDT is expensive due to summations involving as many as eight indices. Among various techniques, CCSDT-n\cite{Noga1987ccsdtn} models are among the earliest ones that retain only the important terms in the triples equation. They are constructed iteratively, approximating triples based on the single and/or double amplitudes, resulting a scaling of $N^{7}$. In the simplest model, known as the coupled cluster singles, doubles, and linearized triple excitation method (CCSDT-1),\cite{Lee1984, Urban1985} only terms with linearized $\hat{T}_{2}$, which contribute the most to triples, are retained in the $\hat{T}_{3}$ equation. Linearized triples, which can improve the MBPT energy to the fourth order, are further used to solve the $\hat{T}_{1}$ and $\hat{T}_{2}$ amplitudes. This is an equivalent approximation to $e^{\hat{T}_{1}+\hat{T}_{2}} + \hat{T}_{3}$ and is often denoted as CCSDT-1a. On the other hand, CCSDT-1b is defined to be a slightly more accurate model, including all the terms in CCSDT-1a, plus the ones involving the product of $\hat{T}_{1}$ and $\hat{T}_{3}$ in the $\hat{T}_{2}$ equation. Additionally, if $e^{\hat{T}_{2}}$ is preserved in the $\hat{T}_{3}$ equation rather than solely $\hat{T}_{2}$, CCSDT-2\cite{Noga1987ccsdtn} can be formulated to include the non-linear term $\hat{T}_{2}^{2}$. Another extended model, CCSDT-3,\cite{Noga1987ccsdtn} approximates $\hat{T}_{3}$ by using $e^{\hat{T}_{1} + \hat{T}_{2}}$ and improves the accuracy compared to CCSDT-2, especially when $\hat{T}_{1}$ amplitudes are large. 

Apart from these iterative models, the investigation of non-iterative models was motivated by reducing the computational cost. Augmented coupled cluster models, including CCSD+T(CCSD)\cite{Urban1985} and CCSD(T)\cite{Raghavachari1989, Stanton1997} methods, use the converged CCSD parameters and introduce triples in a non-interactive manner by adding dominant terms in the fifth-order perturbation expansion of the correlation energy. The correction gained from the triples is evaluated from CCSD $\hat{T}_{2}$ amplitudes in CCSD+T(CCSD), but from both CCSD $\hat{T}_{1}$ and $\hat{T}_{2}$ amplitudes in CCSD(T). Due to the equal treatment of singles and doubles in CCSD(T), a closer approximation to CCSDT is obtained. In practical implementations, the non-iterative methods have only one step that scales at $N^{7}$, while the iterative methods have every iteration scale at $N^{7}$. CCSD(T) is often acknowledged as the gold standard in quantum chemistry, however, it has a deficiency in calculating time-dependent properties due to the non-interactive manner of introducing triples. On the contrary, an approximate coupled cluster singles, doubles and triples (CC3)\cite{Koch1997} model is preferable for such properties. 

CC3 is an iterative model that treats singles uniquely as zero-order parameters for an approximation of orbital relaxation, with triples being correct to the second order. During each iteration, approximate triples are calculated and used for a subsequent calculation of their contributions to the singles and doubles, in order to correct the energy to the fifth order. In this manner, contractions with a scaling of $N^{7}$ occur with the triple amplitudes. Singles and doubles are updated and stored every iteration and returned after convergence, resulting in the same formula for calculating the correlation energy as CCSD.

CC3 can provide comparable results to other iterative and non-iterative approximate triples models for time-independent properties.\cite{Koch1997} More crucially, time/frequency-dependent properties, including response functions, can be properly derived compared to CCSD(T).\cite{Koch1997, Christiansen1995CC3} If considered only in comparison with other iterative models, counting singles as zero-order parameters underscores the greater importance of singles for properties related to the perturbation of an external electromagnetic field and excited states. This makes CC3 an exceptional candidate to be combined with, for example, equation-of-motion methods,\cite{Stanton1993} linear response theory,\cite{Olsen1985, Sekino1984} and real-time methods,\cite{Goings2018, Li2020} in order to involve higher excitations.

Here, we propose the first implementation of the RT-CC3 method, which is built upon our existing RT-CC framework. As stated above, the CC3 method works particularly well for dynamic properties arising from the interaction between the system and the external electromagnetic field. For instance, it excels in calculating induced dipole moments due to its iterative construction and the special treatment of single amplitudes.\cite{Hald2002} Calculations of response functions were soon explored alongside the CC3 model itself and were compared with other iterative models such as CCSDT-1a and CCSDT-1b, where CC3 was found to be favorable.\cite{Christiansen1995CC3} The excitation energy of various molecules was significantly improved by CC3 compared to CCSD and was close to CCSDT. Second-order properties, such as polarizabilities, were also obtained from CC3 linear response functions and showed good agreement with experimental data for the tested molecules.\cite{Hald2003} For these reasons, the RT-CC3 method is desirable for calculating these properties, as it combines advantages from both the real-time formalism and the CC3 model.

The calculation of the absorption spectrum is conducted and compared to the equation-of-motion (EOM) CC results to validate the RT-CC3 method. For higher-order properties, instead of using the Fourier transform of the corresponding time-dependent properties, finite-difference methods\cite{Perrone1975} are used to calculate numerical derivatives to obtain properties including polarizabilities, the $G'$ tensor related to optical rotation, hyperpolarizabilities, etc. This approach was first proposed by Ding et al. in an application of real-time time-dependent density functional theory (RT-TDDFT), allowing response properties to be obtained from a group of propagations with different field strengths.\cite{Ding2013} Without the need for explicit Fourier transform, each propagation does not need to be as long as those for obtaining, for example, absorption spectra, where the resolution of the spectra depends on sufficient propagation time. Additionally, properties related to different orders of response to the same field can be calculated using the same group of propagations. The formalism for calculating different orders of numerical derivatives is relatively simple compared to the derivation of linear, quadratic, or even higher-order response functions. Details about the implementation of the finite difference method and the discussion of the accuracy and stability of the RT-CC3 simulation, along with real-time simulations at other levels of theory, will be presented in the later sections.

Besides response properties, another interesting simulation accessible within the RT framework is the Rabi oscillation,\cite{Knorr1994} which is the oscillation of level population under the influence of an incident light field. A series of discussions on the simulation using RT-TDDFT was published by the Isborn group, focusing on the formulation of the method and addressing false observations due to deficiencies in available exchange functionals.\cite{Habenicht2014, Provorse2015, Provorse2016, Ranka2023} Another application of RT-CC methods on collective Rabi oscillation was conducted by Andreas et al., discussing the differences in quality and stability of the simulation between time-dependent equation-of-motion coupled cluster and time-dependent coupled cluster.\cite{Skeidsvoll2023} In this paper, a simulation of Rabi oscillation is conducted using RT-CCSD, following a discussion on the population of the excited states.



