\section{Conclusions}\label{conc_cc3}
The RT-CC3 method has been implemented with additional single-precision and GPU options. The working equations of RT-CC3 in the closed-shell and spin-adapted formalism are provided, with considerations of optimizing the performance in terms of reducing the number of higher-order tensor contractions. The implementation has been validated through the calculation of the absorption spectrum of \ch{H_{2}O} in both single- and double-precision. Numerical experiments have also been conducted with water clusters to test the computational cost of RT-CC3 simulations. It has been found that the use of GPUs can significantly speed up calculations by up to a factor of 17 due to the computational power they provide for tensor contractions. The acceleration gained from utilizing either GPUs or single-precision arithmetic needs to be observed significantly within a relatively large system (e.g., 72 molecular orbitals). To achieve the theoretical speedup, a much larger system is needed, however, optimization of memory allocation will also need to be taken into account because of the limited memory available on GPUs and the overhead of data migration. With the promising results of our Python implementation, exploring a productive-level code is worthwhile, especially for making the RT-CC3 method, which scales at $N^{7}$, feasible for large system/basis set and/or long RT propagations.  

For the calculation of optical properties, we have demonstrated that RT-CC3 is a feasible tool to obtain dynamic polariazabilities, first hyperpolarizabilities, and the $G'$ tensor with good agreement to LR-CC3 and a reasonable computational cost. The type of applied field, the precision arithmetic and the level of theory were tested with \ch{H_{2}O}. It has been proven through all test cases, including our new RT-CC3 method, that the QRCW can substantially improve curve fitting substantially and requires only two optical cycles for propagation. Especially for first hyperpolarizabilites, the curve of the second-order dipole moments from LRCW calculation has `discontinuities' in some places, leading to a large error and a low R$^{2}$ value for curve fitting. The QRCW is required here to obtain reliable results. The same is found in $G'$ tensor results, where some shifts appear in the curve of the induced magnetic dipole moments from LRCW calculations but not the QRCW ones. Regarding the single-precision calculations, no discrepancy is found in the polarizabilities and $G'$ tensor elements that are associated with the first derivative of electric and magnetic dipole moments, respectively. A significant difference, however, is found in the first hyperpolarizabilities. Although the QRCW can still reduce the error compared to the LRCW, single-precision results remain less accurate compared to double-precision results. These conclusions hold for both RT-CCSD and RT-CC3. 

Additionally, ten-electron systems including \ch{Ne}, \ch{HF}, \ch{H_{2}O}, \ch{NH_{3}} and \ch{CH_{4}} are used to test the calculation of polarizabilities with RT-CCSD, RT-CC3, and particularly TDNOCCD. It has been observed that the accuracy drops significantly when the frequency is closer to the resonance, while for the small frequencies, RT-CC3 matches LR-CC3 and FCI with errors less than 0.1\%. The trend of the error of RT-CC3 is consistent with RT-CCSD for most cases, except for the two values with the highest chosen frequencies of \ch{NH_{3}} and \ch{CH_{4}}. We have shown that the CC3 polarizabilities increases slightly faster when the frequency moves towards resonance, which may lead to a larger error. The TDNOCCD results show that the explicit orbital optimization lowers the polarizability values compared to RT-CCSD, where the only difference in these two methods is the orbital optimization. Compared to RT-CC3, TDNOCCD results are closer to LR-CC3/FCI results when RT-CC3 largely overestimates the results, otherwise, RT-CC3 yields more accurate results. 

Finally, the Rabi oscillation is simulated for \ch{H_{2}} and \ch{C_{2}H_{4}}. For \ch{H_{2}}, the Rabi oscillation between the ground state and the singly-excited state has been successfully reproduced. However, the simulation of the approximate two-level system, $\pi$-$\pi^{*}$ of the double bond, cannot be conducted. The propagation is no longer numerically stable, and the amplitudes diverge when the population of the ground state approaches zero. The reason why RT simulations act differently for the two systems is uncertain. Additional investigation is needed to locate the source of the problem. Using orbital-optimized methods may be a good starting point to determine if stability can be recovered. Using the same applied field but only keeping it on for a short amount of time at the beginning of the propagation, the transition to doubly-excited state of \ch{H_{2}} is observed from the absorption spectrum, although the RT-CC framework is built upon a RHF reference. Furthermore, the field is applied for one optical cycle for \ch{LiH}, with various field strengths. With the increasing field strength, the transitions can be found in the absorption spectrum, not only for ground state to ground state transitions, but also for excited state to excited state transitions. As stated earlier, although only RT-CCSD simulations were carried out for this section, RT-CC3 can be used in the same fashion if needed. 
