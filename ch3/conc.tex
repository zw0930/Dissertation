%conc.tex%
\section{Conclusions} \label{conc}

In this work, we have explored several approaches for improving the
efficiency of the real-time coupled cluster singles and doubles method.
Through a number of numerical experiments on absorption spectra of small
clusters of water molecules, we have found that lowering the arithmetic
representation of the wave function from double- to single-precision yields
negligible differences in the resulting spectra, but speeds up the
calculations by nearly a factor of two compared to the conventional
double-precision implementation.  We have additionally found that migration
of the data and the corresponding tensor contractions from CPU to GPU
utilizing the the PyTorch framework produces a further overall speedup of a
factor of 14.  Based on the rapidly growing computational power of GPU
hardware and their supporting software ecosystem, we intend to carry out
further investigation and optimization of our GPU implementation for
calculations on larger molecular systems. 

We have also investigated a variety of numerical integration schemes for
improving the stability and efficiency of the RT-CC approach, especially
focusing on adaptive integrators that can adjust the step size during the
time-propagation.  In particular, we demonstrate that the Cash-Karp
integrator, which uses an estimate of the local error in each iteration, can
accordingly adjust the time-step to optimize the simulation in terms of
both computing time and numerical stability.  However, for very strong
external fields, such as a narrow Gaussian envelope or delta-pulse, both of
which are commonly used in such simulations, we find that even a
straightforward mixed-step integrator based on the fourth-order Runge-Kutta
algorithm is capable of providing a stable propagation provided a small
enough step size is used in for the duration of the field.  Such an
algorithm should be quite favorable for such calculations with intense, but
narrow laser pulses since it enables the existing RT-CCSD method to be
generally used without any substantial modification to the algorithm or
excessively increased computational cost. While further tests are required to determine the generality and robustness of
this approach, the current results are encouraging.

