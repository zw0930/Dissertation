\section{Introduction} \label{intro}

Although quantum chemical models for the properties of stationary states have
seen great advances over the last 60 years, both in terms of accuracy and
computational efficiency, the non-equilibrium character of time-dependent
Hamiltonians (e.g., in the presence of an external, oscillating
electromagnetic field) requires approaches based on the time-dependent
Schr\"odinger equation.\cite{McCullough1969, Kosloff1988, Li2020}
While the most common techniques take a perturbational approach and treat
spectroscopic responses in the frequency domain, thereby carefully avoiding
the often-expensive time-propagation of the wave function, explicitly
time-dependent methods have a number of important advantages over their
perturbative counterparts. First, time-dependent methods allow straightforward
connections to experimental conditions, such as fine-tuning the shape,
duration, and intensity of the external fields.  Second, such methods yield
spectroscopic properties across a wide range of frequencies via Fourier
transformation of, e.g., the time-dependent electric-dipole moment, rather
than a relatively narrow window of frequencies produced by response
techniques.  Third, with careful propagation algorithms, time-dependent
methods can permit simulation of more intense external fields than
perturbation theory approaches.  Finally, the ample resources available from
modern numerical mathematics and computer science may be brought to bear to
improve the stability and efficiency of the time propagation itself.

To exploit these advantages, a wide range of real-time methods have been
explored over the last 30 years based on a variety of approximate solutions to
the electronic Schr\"odinger equation, including Hartree-Fock
(HF)\cite{Li2005}, density-functional theory (DFT),\cite{Repisky2015, Goings2018}
configuration interaction (CI),\cite{Klamroth2003,Schlegel2007} and
coupled-cluster (CC)\cite{Huber2011, Nascimento2017, Kristiansen2020}
approaches.  Among these, real-time DFT (RT-DFT) is the most widely used for
spectroscopic applications across a range of fields from biochemistry to solid
state physics where the target systems are relatively large.\cite{Sanchez2010,
Kolesov2016, Provorse2016, Bruner2016, Goings2016} The theory behind real-time
methods is continuously under development, however, and the same shortcomings
of systematic convergence and limited robustness that apply to ground-state
density-functionals also apply to RT-DFT.  This motivates researchers to
investigate higher-level methods.

Real-time coupled cluster (RT-CC) methods, in particular, can achieve
exceptionally high accuracy in many cases\cite{Bartlett2010,
Crawford2000} and have been explored in the context of real-time
simulations for some years.\cite{Huber2011, Pedersen2019, Kristiansen2020,
Pedersen2020,
Nascimento2016linear, Nascimento2017, Nascimento2019, Koulias2019, Vila2020,
Park2019, Park2021, Cooper2021} However, RT-CC approaches also suffer from
the same affliction as that of their time-independent counterparts,
\textit{viz.}, high-degree polynomial scaling with the size of the molecular
system.  For ground-state coupled cluster theory, techniques such as local
correlation, \cite{Pulay1983, Hampel1996, Neese2009, Schutz2001,Yang2012,
Russ2004, McAlexander2012, Crawford2019} fragmentation,\cite{Gordon2012,
Epifanovsky2013, Li2004, Li2009} tensor decomposition,\cite{Kinoshita2003,
Koch2003, Hohenstein2012, Schutski2017, Parrish2019, Pawlowski2019} and others
have been developed to permit applications to larger molecular systems than
conventional implementations allow.  However, while such methods are
potentially transferable to the corresponding time-dependent approaches, they
have yet to be exploited to reduce the computational cost of RT-CC.

In addition to the development of more compact representations of the
time-dependent wave function, the construction of the differential equations
that represent the time-dependency of the relevant properties, as well as the
choice of numerical integration algorithm can substantially affect the
efficiency and/or the stability of the calculation.  For example, DePrince and
Nascimento\cite{Nascimento2016,Nascimento2017} introduced left and right coupled
cluster dipole functions within the equation-of-motion (EOM-CC) framework for
calculating accurate linear absorption spectra across a wide frequency range.
They demonstrated that propagating either the left- or right-hand dipole
functions yielded the same result, thus reducing the computational cost by a
factor of two compared to propagating both left- and right-hand coupled cluster
wave functions. In 2016, Lopata\cite{Bruner2016} and coworkers accelerated
RT-DFT calculations of broadband spectra by applying Pad{\'e} approximants to
the Fourier transforms. Note that they gained a five times shorter simulation
time by obtaining rapid convergence of spectra from Pad{\'e} approximants,
while the technique has no dependence on the level of theory. In 2019, Pedersen and
Kvaal\cite{Pedersen2019} reported symplectic integrators such as Gauss-Legendre
can provide stable implementations across long propagation times even with
relatively strong external fields. In subsequent work, Pedersen, Kvaal, and
co-workers\cite{Kristiansen2020} used orbital-adaptive time-dependent coupled
cluster doubles (OATDCCD) to improve the stability of their RT-CC
implementation even when strong external fields result in a non-dominant
electronic ground state. For the application to core
excitation spectra, Bartlett and coworkers\cite{Park2019,Park2021}
compared time-independent (TI) EOM-CC and time-dependent (TD) EOM-CC --- as
well as contributions beyond the dipole approximation --- and concluded that
TD-EOM-CC can provide accurate spectra for both core and valance spectra. In
addition, in 2021, Li, DePrince, and co-workers\cite{Cooper2021} applied the short
iterative Lanczos integration to the time-dependent (TD) EOM-CC method for a
more efficient calculation of K-edge spectra. 

In this paper, we consider alternative numerical approaches aimed at reducing
the cost of RT-CC calculations.  In most quantum chemical programs,
numerical parameters are typically computed and stored using binary
representations translating to approximately 15 decimal digits of (double)
precision.  However, pioneering studies by Yasuda,\cite{Yasuda2008} Martinez and
co-workers,\cite{Ufimtsev2008,Ufimtsev2008quantum,Luehr2011} Aspuru-Guzik and
co-workers,\cite{Vogt2008} DePrince and Hammond,\cite{DePrince2011} Asadchev and Gordon\cite{Asadchev2012}, and Krylov
and co-workers\cite{Pokhilko2018} have demonstrated that single-precision
arithmetic in which the binary representation supports ca.\ seven decimal
digits, is sufficient --- and more cost effective --- for many applications.
Based on the success of this previous work, we have explored the use of
single-precision arithmetic in the context of RT-CC codes, particularly
for the simulation of linear absorption.
Additionally, we report a single-precision RT-CC implementation for which we
obtain significant further speed-up utilizing the parallel architecture of
graphical processing units (GPUs).

Finally, we explore a range of numerical integrators for solving the RT-CC
differential equations for time-dependent properties.  Generally, there are
three main types of such integrators: explicit, adaptive (or embedded), and
implicit.  Explicit integrators are so named because they take into account
only the output of the previous time step when calculating the results of the
current step, whereas adaptive integrators can adjust the step size at each
iteration with specialized algorithms to control the local error. Finally,
implicit integrators take in account the outputs of both the previous step and
the current step when determining the results for the next step, an approach
that is often more expensive due to the required iterative algorithm.
The Runge-Kutta (RK) family of integrators\cite{Butcher1996} includes all
three types and is commonly used for solving the initial value problem
associated with the Schr\"odinger equations-of-motion.  We have built a library of RK
integrators using all three types that are compatible with the RT-CC
algorithm, and here we compare their performance.  Moreover, inspired by
conventional adaptive integrators, we examine a mixed-step-size formalism to
customize the propagation on the fly, depending on whether the external field
is on or off.

%A stable and trackable
%time series of solutions is obtained for the cases with strong external fields
%without adding significant computational cost. Details will be discussed in
%the following sections.

