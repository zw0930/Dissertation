\chapter{Conclusions} \label{conc}
In this work, we have explored how molecules interact with external electromagnetic fields using real-time methods. The study began by establishing a framework that can be generally applied to RT-CC methods with varying levels of accuracy. Through this, we attempted a GPU implementation, single-precision calculations, and various numerical integration techniques to enhance the efficiency and numerical stability of the method. Additionally, we investigated the implementation of a higher level of theory, CC3, within the RT framework. A range of optical properties were calculated and then compared to results obtained from conventional linear response theory. Providing context and an overview of this topic, the background and objectives are summarized in Chapter 1. The relevant theoretical frameworks, including coupled cluster methods, response theory, and real-time methods, are presented in Chapter 2.

Chapter 3 focused on enhancing the computational efficiency of RT-CC methods in several aspects. Firstly, single-precision arithmetic was applied and tested for the calculation of the absorption spectrum of the water molecule, considering both computational speed and memory usage. Random noise of varying magnitudes was introduced to the double-precision results to assess the impact of precision level. No significant difference was observed between single- and double-precision absorption spectra in the water cluster test cases. The calculation was accelerated by a factor of 1.5 to 2 for a moderately larger system size.

Secondly, a GPU implementation was developed to accelerate tensor contractions, which represent the most resource-intensive part of the calculation. To minimize modifications to the existing CPU implementation, PyTorch was utilized to replace NumPy for tensor operations. The contractions were performed using {\tt opt\_einsum} with PyTorch backend. Numerical experiments demonstrated that the speedup from GPUs increases with system size. For the water tetramer test case, a speedup of a factor of 14 was achieved. This speedup is expected to continue growing for larger systems until it reaches the memory limit of GPUs.

Lastly, a family of Runge-Kutta integrators for the explicit propagation of wave function parameters in RT methods was studied and implemented. The explicit integrator was initially introduced and used to convey the fundamental concepts of numerical integration. Adaptive integrators, capable of adjusting the step size based on the local error at each time step, were further tested for potential speedup while maintaining numerical stability. Results showed that adaptive integrators can achieve a speedup compared to explicit integrators of the same order, assuming there are no abrupt changes in the trajectory.

For strong-field calculations, a mixed-step-size approach inspired by adaptive integrators was developed. This approach targets the rapidly increasing coupled cluster amplitudes when the field is active. A small step size, such as $10^{-5}$ au, is chosen for that period, while a larger step size, such 0.01 au, is used for the remainder of the propagation. This approach effectively stabilizes the calculation with minor overhead, especially when a thin Gaussian pulse is used as the external field.

In Chapter 4, we have formulated and implemented the RT-CC3 method to achieve a higher level of accuracy in comparison to the prevailing RT-CCSD method. Within the CC3 model, singles are designated as zeroth-order in perturbation theory and treated as parameters for orbital relaxation. Triples are computed in each iteration, with updated singles and doubles, and are employed to calculate their contributions to the new amplitude residuals.

The RT-CC3 absorption spectrum of the water molecule was computed in both single- and double-precision. It was demonstrated that these two levels of precision yield identical spectra that align with the EOM-CC3 results. Furthermore, the performance of both GPU and CPU implementations was evaluated using the same water cluster test cases introduced in Chapter 3. The results clearly indicated that the runtime of RT-CC3 calculations was notably longer than that of RT-CCSD, due to the $N^{7}$ scaling. However, through the utilization of GPUs, a significant speedup of up to a factor of 17 was achieved.

Beyond the absorption spectrum, RT-CC3 has been extended to compute higher-order properties and to be compared against other RT methods. Dynamic optical properties, encompassing polarizabilities, first hyperpolarizabilities, and the $G'$ tensor, were obtained using the finite difference method to calculate the first and second-order derivatives of the time-dependent induced electric and magnetic dipole moments. In comparison to their double-precision counterparts, single-precision calculations were also conducted.

From the numerical tests on the water molecule, it was observed that the quadratic ramped continuous wave (QRCW), which ensures an adiabatic switch-on, yielded more accurate results and higher curve fitting quality compared to the linear ramped continuous wave (LRCW). Concerning polarizabilities, the single- and double-precision results were identical and differed from the linear response (LR) results by less than 0.1\%. For first hyperpolarizabilities, a discrepancy between single- and double-precision arithmetic was observed, with single-precision results being less accurate. The results showed larger errors, up to 2.56\% for single-precision and 0.93\% for double-precision, which stemmed from the higher-order dipole moments compared to polarizabilities. Results of the $G'$ tensor were similar to polarizabilities, exhibiting no significant difference between single- and double-precision results and maintaining a high-quality curve fitting.

Furthermore, RT-CCSD and RT-CC3 polarizabilities were compared to time-dependent nonorthogonal orbital-optimized coupled cluster doubles (TDNOCCD) results for five ten-electron systems. It was demonstrated that the accuracy of RT results decreased as the corresponding frequencies approached resonance. The orbital optimization within TDNOCCD led to smaller polarizability values across various systems and frequencies, in comparison to RT-CCSD and RT-CC3.

With our RT implementation, we have also conducted simulations involving electron dynamics, including Rabi oscillations and transitions to excited states. The Rabi oscillation of the hydrogen molecule was successfully reproduced by applying an oscillatory field with the excitation energy from the ground state to the singly-excited state as the frequency. The population of the ground state, represented by the square of the autocorrelation function of the ground state and the state vector at each time step, fluctuated between 0 and 1. The frequency, position of the wave crest, and trough matched the characteristics of a general Rabi oscillation.

A similar simulation was attempted for the Rabi oscillation of an approximate two-level system, $\pi$-$\pi^{*}$ of \ch{C_{2}H_{4}}. However, the propagation encountered issues around the time when the ground state was depleted. While this behavior might be reasonable considering our RT methods are built upon a single reference, the distinct difference compared to the simulation of \ch{H_{2}} remains unexplained with the current results.

In addition to the Rabi oscillation, we observed transitions between excited states for the \ch{H_{2}} and \ch{LiH} test cases, particularly when the frequency of the external field was selected to match the $\sigma$-$\sigma^{*}$ transition. As the field strength increased, a corresponding enhancement in the relative intensities of the excited state to excited state transitions was demonstrated in the absorption spectra.

Overall, this work has demonstrated the process of obtaining optical properties using RT-CC methods and enhancing the efficiency of these calculations. Through the development of the RT-CC framework, the established infrastructure, GPU implementation, and distinct precision arithmetics are readily applicable to various levels of theory with minimal adjustments. Simulations have confirmed that our RT-CC implementation offers a functional and user-friendly tool for computing frequency-dependent optical properties in the time domain.

For future investigations, the potential development of a production-level code incorporating the GPU implementation could expand the capacity to compute optical properties for larger systems. Additionally, collaborative efforts with experimental scientists are anticipated, as the parameters of the external field can be tailored to match experimental setups within the existing RT framework.